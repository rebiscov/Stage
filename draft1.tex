\documentclass[10pt,a4paper]{article}
\usepackage[utf8]{inputenc}
\usepackage{amsmath}
\usepackage{amsfonts}
\usepackage{amssymb}
\usepackage{graphicx}
\title{Prise en compte du délai de changement de vitesse: première idée}
\begin{document}
\maketitle

\section{But}
On considère toujours le problème où l'on cherche à optimiser la
consommation d'un processeur mais cette fois, il faudra un temps
$\tau$ pour passer d'une vitesse $v_i$ à une vitesse $v_j$ si $v_i\neq
v_j$

\section{Calcul des vitesses}
Il faudra recalculer ``l'enveloppe convexe'' des vitesses pour que
celle-ci prenne en compte ce délai. Ensuite, on ne va plus considérer
$J_t^*$ (la consommation optimale) mais $J_t^{v_i}$ qui est la
consommation optimale sachant qu'à $t-1$, on allait à la vitesse $v_i$.

\section{Formule triviale}

\begin{equation}
  \resizebox{1.3\hsize}{!}{$
    \begin{array}{ll}
      J_T^{v_i}(w)=0 & \forall v_i, w\\
      J_{t-1}^{v_i}(w) = \min
      \begin{cases} 
        \min_{v_j\neq v_i, (1-\tau)v_j\geq w(1)} [\tau
        \times L+(1-\tau)c(v_j)+\sum_{w'\in\mathcal{W}}J_t^{v_j}(w')\times
        P(w,w',v_j)]\\
        c(v_i)+\sum_{w'\in\mathcal{W}}J_t^{v_i}(w')\times P(w,w',v_i)]
        \text{ si } v_i\geq w(1)

      \end{cases} & \text{ si } 1\leq t-1 \leq T-1 \\
      
      J_0^{v_i}(w) = \min_{v_j\geq
      w(1)}c(v_j)+\sum_{w'\in\mathcal{W}}J_1^{v_j}(w')\times
      P(w,w',v_j)] & \forall v_i,w 

    \end{array}
  $}
\end{equation}

\subsection{Faiblesses}

Le problème avec cette méthode est que le changement de vitesse ne
peut avoir lieu qu'au début d'un intervalle de temps $[t, t+1[$. Or on
peut très bien imaginer un changement de vitesse à la fin d'un
intervalle de temps. Cependant, ce n'est pas si grave car on ne peut
changer de vitesse qu'aux instants entiers.

\subsection{Avantages}

Si la vitesse maximale est assez grande, il y a toujours une solution.

\section{Compléxité}

Il faut multiplier la complexité par $|\mathcal{V}|$, on obtient alors
une complexité en $\mathcal{O}(T\times|\mathcal{V}|^2 \times
S\times\Delta\times C(S,\Delta))$, du moins, si on suppose qu'on reste
dans les entiers, c'est à dire $(1-\tau)v_i\in\mathbb{N}$, ce qui est
peu probable. Néanmoins, si $\tau\in\mathbb{Q}$, alors
$(1-\tau)v_i\in\mathbb{Q}$, ainsi, on peut multiplier les vitesses par
PPCM$(\{\text{den}((1-\tau)v_i)\})=p$. De même il faut modifier les
fonctions remaining work, $w_t\rightarrow p\times w_t$.


\end{document}
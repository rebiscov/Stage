\documentclass[10pt,a4paper]{article}
\usepackage[utf8]{inputenc}
\usepackage{amsmath}
\usepackage{amsfonts}
\usepackage{amssymb}
\usepackage{graphicx}
\newcommand{\Q}{\mathbb{Q}}
\newcommand{\N}{\mathbb{N}}
\newcommand{\W}{\mathcal{W}}
\newcommand{\V}{\mathcal{V}}
\title{Algo: première implémentation}
\begin{document}
\maketitle

\section{But}
On va modéliser le changement de vitesse comme étant un coût
supplémentaire. On garde presque la même formule, on va juste ajouter
un coût $f(v_i,v_j)$ tel que:\\
$f(v_i,v_j)=
\begin{cases}
  0 & \text{si } v_i=v_j \\
  f(v_i,v_j)>0 & \text{sinon}
\end{cases}
$

L'enjeu va être de trouver la bonne fonction $f$, sinon la formule
change peu, mis à part qu'elle prend en compte la vitesse
précédente. On se retrouve alors avec la formule:
\begin{equation}
  J_t^{v_i} =
  \begin{cases}
    0 & \text{si } t=T\\
    \min_{v\in\V}
    c(v)+f(v_i,v)+\sum_{w\in\W'}J_{t+1}^v\times\mathbb{P}(w,w',v) & 1\leq
    t\leq T-1\\
        \min_{v\in\V}
    c(v)+\sum_{w\in\W'}J_{t+1}^v\times\mathbb{P}(w,w',v) & \text{si }t=0

  \end{cases}
\end{equation}
\end{document}
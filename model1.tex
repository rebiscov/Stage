\documentclass[10pt,a4paper]{article}
\usepackage[utf8]{inputenc}
\usepackage{amsmath}
\usepackage{amsfonts}
\usepackage{amssymb}
\usepackage{graphicx}
\newcommand{\Q}{\mathbb{Q}}
\newcommand{\N}{\mathbb{N}}
\newcommand{\W}{\mathcal{W}}
\newcommand{\V}{\mathcal{V}}
\title{Prise en compte du délai de changement de vitesse: première idée}
\begin{document}
\maketitle

\section{Introduction}

Dans cette première approche, nous allons considérer que
$\tau\in\Q$ et que lorsque que l'on passe de la vitesse $v_i$
à $v_j$ avec $v_i\neq v_j$, alors on ne travaille pas pendant un temps
$\tau$ et on travaille durant $1-\tau$.

\section{Mise en oeuvre}

\subsection{Se ramener au cas entier}

Si il y a changement de vitesse de $v_i$ à $v_j$, $v_i\neq v_j$ à
l'instant $t\in\N$, alors le travail effectué durant l'intervalle
$[t,t+1[$ sera $(1-\tau)v_j$ ce qui n'est pas forcément entier. Or,
$\tau\in\Q$ donc $(1-\tau)v_j\in\Q $. Ainsi, on peut écrire
$(1-\tau)v_j=\frac{p_j}{q_j}$ avec $p_j\wedge q_j=1$. Soit
$p=\text{PPCM}\{q_j\}$, $p\in\N$. Alors $\forall i, p\times v_i\in\N$
et $p\times(1-\tau)v_i\in\N$, on fait alors $\V:=p\times\V$. On a
changé les vitesses, désormais les vitesses permettent de faire $p$
fois plus de travail qu'auparavant, il faut donc que les ``jobs''
entrant aient des ``workloads'' $p$ fois plus important. Ainsi, on
remplace tous les jobs $(a_i,s_i,d_i)$ par $(a_i, p\times s_i, d_i)$,
de plus, on a un nouveau seuil maximum: $S'=p\times S$.
\subsection{Quelques considérations}

On considère le problème online. On va alors supposer que 
l'on peut changer de vitesse seulement au début d'un intervalle de
temps $[t,t+1[$. Cela veut dire que lorsque que l'on change de
vitesse, la période de transition se fait dans $[t,t+\tau[$. On
considéra de plus une consommation résiduelle ``leakage'' $L$ lorsque
l'on change de vitesse durant le temps $\tau$.


\section{Formule}


\begin{equation}
  \resizebox{1.3\hsize}{!}{$
    \begin{array}{ll}
      J_T^{v_i}(w)=0 & \forall v_i, w\\
      J_{t-1}^{v_i}(w) = \min
      \begin{cases} 
        \min_{v_j\neq v_i, (1-\tau)v_j\geq w(1)} [\tau
        \times L+(1-\tau)c(v_j)+\sum_{w'\in\mathcal{W}}J_t^{v_j}(w')\times
        P(w,w',v_j)]\\
        c(v_i)+\sum_{w'\in\mathcal{W}}J_t^{v_i}(w')\times P(w,w',v_i)]
        \text{ si } v_i\geq w(1)

      \end{cases} & \text{ si } 1\leq t-1 \leq T-1 \\
      
      J_0^{v_i}(w) = \min_{v_j\geq
      w(1)}c(v_j)+\sum_{w'\in\mathcal{W}}J_1^{v_j}(w')\times
      P(w,w',v_j)] & \forall v_i,w 

    \end{array}
  $}
\end{equation}

\section{Commentaires}

\subsection{Complexité}

La solution est simple car l'algorithme change peu. Néanmoins, il faut
recalculer $J_t^*$ pour chaque vitesse, ainsi on multiplie la
complexité par $|\V|$. Celle-ci devient alors: \[\mathcal{O}(T\times|\mathcal{V}|^2 \times
S'\times\Delta\times C(S',\Delta))=\mathcal{O}(T\times|\mathcal{V}|^2
\times p\times S\times\Delta\times C(p\times S,\Delta))\]
On n'oublie pas de prendre en compte le nouveau seuil $S'$.

\subsection{Avantages}

Le problème change peu, la transformation est simple.

\end{document}
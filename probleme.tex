\documentclass[10pt,a4paper]{article}
\usepackage[utf8]{inputenc}
\usepackage{amsmath}
\usepackage{amsfonts}
\usepackage{amssymb}
\usepackage{graphicx}
\title{Optimisation avec incertitudes: application à la gestion énergétique}
\begin{document}
\maketitle

\section{Le problème}
Un processeur va à la vitesse $v(t)\in \mathbb{N}$. La consommation de
ce processeur est $c(v)$ qui est souvent une fonction convexe, on la
prendra souvent de la forme $c(v)=\mbox{cst}\times v^\alpha$ avec
$\alpha>1$\\
Le processeur doit effectuer des ``jobs'' de la forme $J_i=(a_i, s_i,
d_i)$.
\begin{itemize}
\item $a_i$ est l'instant où le travail débute
\item $s_i$ est le travail à effectuer
\item $d_i$ est la deadline relative, c'est à dire qu'à $a_i+d_i$, le
  travail doit être terminé.
\end{itemize}
Supposons que $t\in [|0,T|]$
On note $w_t(u)$ le travail restant. $J_t(w)=$ consommation entre $t$.
et $T$.\\

Notons $J_t^*(w)$ la consommation optimale, on a alors\\

\begin{equation}
  \begin{array}{l}
    J_T(w)=0~\forall w\in W\\
    J_{t-1}^*(w)=\min\limits_{v\in V, v\geq w(1)}[c(v)+\sum_{w'\in
    W}J_t^*(w')\times\mathbb{P}(w,w',v)]
    
  \end{array}
\end{equation}
À chaque instant on a une probabilité qu'un job arrive tel que:
\begin{equation}
  \mathbb{P}(s=\sigma, d=\delta)=p(\sigma,\delta)  
\end{equation}

On a de plus la formule:
\begin{equation}
  w_{t+1}(x)=T(w_t(x)-v)+\sigma H_{\delta}
\end{equation}

avec $T(f(t))=f(t+1)$, on prendra $\mathbb{P}(w, w', v)=p(\sigma,
\delta)$.


\section{Changement de vitesse}
Sur la courbe ``consommation = $f(v)$'', on prend l'enveloppe
convexe. En effet, on peut ``combiner'' les vitesses.\\
Si je consomme 0 à la vitesse 0, 3 à la vitesse 1, et 4 à la vitesse
2, pour produire 1 de travail, il est plus intelligent d'aller à la
vitesse 2 pendant 0.5 s car je consommerai 2 plutôt que 3 à la vitesse
1 pour le même travail.

\section{TODO}
But: maintenant on suppose que que le changement de vitesse se fait
pendant un temps $\tau$ durant lequel la processeur ne produit
rien. On pourra aussi s'intéresser à un changement de vitesse
progressif, modélisable par une fonction $C^1$

\section{Début de de solution évidente}


\end{document}
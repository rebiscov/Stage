\documentclass{beamer}
%
% Choose how your presentation looks.
%
% For more themes, color themes and font themes, see:
% http://deic.uab.es/~iblanes/beamer_gallery/index_by_theme.html
%
\mode<presentation>
{
  \usetheme{default}      % or try Darmstadt, Madrid, Warsaw, ...
  \usecolortheme{default} % or try albatross, beaver, crane, ...
  \usefonttheme{default}  % or try serif, structurebold, ...
  \setbeamertemplate{navigation symbols}{}
  \setbeamertemplate{caption}[numbered]
} 

\usepackage[francais]{babel}
\usepackage[utf8x]{inputenc}

\usepackage{tikz}
\usepackage{stmaryrd}
\usepackage{amsthm}
\usepackage{algorithm}
\usepackage{algorithmic}

\DeclareMathOperator*{\argmin}{arg\,min}
\DeclareMathOperator{\lcm}{lcm}

\newcommand{\N}{\mathbb{N}}
\newcommand{\V}{\mathcal{V}}
\newcommand{\U}{\mathcal{U}}
\newcommand{\Sp}{\mathcal{S}}
\newcommand{\T}{\mathbb{T}}
\newcommand{\W}{\mathcal{W}}
\newcommand{\Q}{\mathbb{Q}}
\newcommand{\R}{\mathbb{R}}

\title[Stage Inria]{Optimisation avec incertitudes: application à la gestion
  énergétique}
\author{Vincent RÉBISCOUL}
\institute{École Normale Supérieure de Lyon}
\date{Mardi 5 Septembre 2017}

\begin{document}

\begin{frame}
  \titlepage
\end{frame}


\begin{frame}{Plan\textsc{}}
  \tableofcontents
\end{frame}

\section{Introduction}
\begin{frame}{Introduction}

  Nous avons un processeur et nous voulons optimiser sa consommation.

  \begin{itemize}
  \item Coût de l'énergie
  \item Coût du refroidissement
  \item Pollution
  \item Amélioration de la longévité de l'équipement
  \end{itemize}
  
  \begin{figure}
    \centering
    \includegraphics[scale=0.1]{fan.jpg}
  \end{figure}
\end{frame}

\section{Premier modèle}

\subsection{Les "jobs"}

\begin{frame}{Jobs}
  $j_i=(a_i,s_i,d_i)\in\N^3$
  \begin{itemize}
  \item $a_i$ le temps d'arrivée
  \item $s_i$ la charge de travail
  \item $d_i$ le temps donné pour effectuer le travail
  \end{itemize}

  $\forall j_i$, $d_i\leq\Delta$ et
  $s_i\leq S$
  
  \begin{figure}
  \centering
  \caption{Exemple de ``jobs''}
  \label{fig:jobs}  
  \begin{tikzpicture}[scale=0.7]
    \draw[help lines] (0,0) grid (10,5);
    \draw[<->] (0,5.5) -- (0,0) -- (10.5,0);
    \foreach \y in {0,1,...,5} \draw (-0.25,\y) node {\y};
    \foreach \x in {0,1,...,10} \draw (\x, -0.25) node {\x};
    \draw (11,0) node {time};
    \draw (0,6) node {workload};
    \draw[|-|,cyan,thick] (1,2) -- (3,2);
    \draw[cyan] (2,2.3) node {$j_1$};
    \draw[|-|,red,thick] (2,3) -- (6,3);
    \draw[red] (4,3.3) node {$j_2$};
    \draw[|-|,brown,thick] (4,4) -- (9,4);
    \draw[brown] (6.5,4.3) node {$j_3$};
    \draw[|-|,purple,thick] (5,1) -- (10,1);
    \draw[purple] (7.5,1.3) node {$j_4$};
  \end{tikzpicture}
\end{figure}
\end{frame}

\subsection{Quelques définitions}

\begin{frame}{Définitions}
  \begin{itemize}
  \item $\V\subset\N$ est l'ensemble des vitesses
  \item $T$ est le temps de travail
  \item $D(t)$ la fonction du travail à faire
  \item $A(t)$ la fonction du travail arrivé
  \item $\phi_w(\sigma,\delta)$
  \item $\W$ l'ensemble des fonctions de travail restant
  \end{itemize}

  \begin{figure}
    \centering
    \includegraphics[scale=.15]{dico.jpg}
  \end{figure}

\end{frame}

\begin{frame}
  \begin{figure}
    \centering
    \begin{tikzpicture}[scale=0.75]
      \draw[help lines] (0,0) grid (10,10);
      \foreach \x in {0,1,...,10} \draw (\x,-0.25) node {\x};
      \foreach \y in {0,1,...,10} \draw (-0.25,\y) node {\y};
      \draw[<->] (0,10.5) -- (0,0) -- (10.5,0);
      \draw[green,ultra thick] (0,0) -- (3,0) -- (3,2) -- (6,2) -- (6,5)
      -- (9,5) -- (9,9) -- (10,9) -- (10,10);
      \draw[green] (6.5,2.5) node {$D(t)$};
      \draw[red, ultra thick] (0,0) -- (1,0) -- (1,2) -- (2,2) -- (2,5)
      -- (4,5) -- (4,9) -- (5,9) -- (5,10) -- (10,10);
      \draw[red] (3.5,5.5) node {$A(t)$};
      \draw[blue,thick] (0,0) -- (1,0) -- (3,2) -- (4,4) -- (5,5) -- (6,7) -- (7,7)
      -- (8, 9) -- (9,9) -- (10,10);
      \draw[blue] (6.5, 7.5) node {$e(t)$};
    \end{tikzpicture}
    \caption{Les fonctions $D(\cdot)$ et $A(\cdot)$}
    \label{fig:D}
  \end{figure}  
  
\end{frame}

\subsection{La fonction de travail restant}

\begin{frame}
  \begin{figure}
    \centering
    \begin{tikzpicture}[scale=0.65]
    \draw[help lines] (0,0) grid (5,5);
    \draw[<->] (0,5.5) -- (0,0) -- (5.5,0);
    \draw (6,0) node {time};
    \draw (-2,4) node {workload};
    \foreach \x in {0,1,...,5} \draw (\x, -0.25) node {\x};
    \foreach \y in {0,1,...,5} \draw (-0.25,\y) node {\y};
    \draw[purple,ultra thick] (0,0) -- (1,0) -- (1,2) -- (4,2) --
    (4,5) -- (5,5);
    \draw[green,dashed,ultra thick,|-|] (0.90, 0) -- (0.90,2);
    \draw[green] (3.5, 3.5) node {$j_2$};
    \draw[green,dashed,ultra thick,|-|] (3.90, 2) -- (3.90,5);
    \draw[green] (0.5, 1.5) node {$j_1$};    
  \end{tikzpicture}
  \begin{tikzpicture}[scale=0.65]
    \draw[help lines] (0,0) grid (10,5);
    \draw[<->] (0,5.5) -- (0,0) -- (10.5,0);
    \foreach \y in {0,1,...,5} \draw (-0.25,\y) node {\y};
    \foreach \x in {0,1,...,10} \draw (\x, -0.25) node {\x};
    \draw (11,0) node {time};
    \draw (-2,4) node {workload};
    \draw[|-|,cyan,thick] (1,2) -- (3,2);
    \draw[cyan] (1.5,2.3) node {$j_1$};
    \draw[|-|,red,thick] (2,3) -- (6,3);
    \draw[red] (4,3.3) node {$j_2$};
    \draw[|-|,brown,thick] (4,4) -- (9,4);
    \draw[brown] (6.5,4.3) node {$j_3$};
    \draw[|-|,purple,thick] (5,1) -- (10,1);
    \draw[purple] (7.5,1.3) node {$j_4$};
    \draw[green,dashed,thick] (2,0) -- (2,5);
  \end{tikzpicture}
    \caption{La fonction de travail restant}
    \label{fig:remaining_work}
  \end{figure}
\end{frame}

\subsection{Politique}
\begin{frame}{Politique}
  \begin{block}{Politique}
    Un politique $\pi(\cdot,\cdot)$ est une fonction tel que
    \[
      \pi:\W\times \{0,\dots,T\}\rightarrow\V
    \]
  \end{block}

  \begin{block}{$J_t(w)$}
    $J_t(w)$ est la consommation moyenne étant donné une politique
    $\pi(\cdot,\cdot)$ à l'instant $t$ et à l'état $w$.\\
    \[
      J_t(w)=\mathbb{E}\sum_{s=t}^Tc(\pi(w_s,s))
    \]
    avec $w_t=w$
  \end{block}
  
\end{frame}

\subsection{L'algorithme}

\begin{frame}{L'algorithme}
  \begin{algorithm}[H]
    \caption{Algorithme de programmation dynamique pour trouver la
      politique optimale}
    \label{alg:dpa}
    \begin{algorithmic}
      \STATE $t\leftarrow T$
      \FORALL{$w\in\W$}
      \STATE $J_T^*(w)\leftarrow 0$
      \ENDFOR
      \WHILE{$t\geq 1$}
      \FORALL{$w\in\W$}
      \STATE $J_{t-1}^*(w)\leftarrow
      \min\limits_{v\in\V,v\geq
        w(1)}\left(c(v)+\sum\limits_{w'\in\W}P_v(w,w')\times
        J_t^*(w')\right)$
      \STATE $\pi_{t-1}^*(w)\leftarrow
      \argmin\limits_{v\in\V,v\geq
        w(1)}\left(c(v)+\sum\limits_{w'\in\W}P_v(w,w')\times
        J_t^*(w')\right)$
      \ENDFOR
      \STATE $t\leftarrow t-1$
      \ENDWHILE
      \RETURN $\pi^*$
    \end{algorithmic}
  \end{algorithm}
  
\end{frame}

\begin{frame}
  \begin{figure}
    \centering
    \includegraphics[scale=.4]{example_simu.png}
    \caption{Simulation d'un politique optimale}
    \label{fig:exemple}
  \end{figure}
\end{frame}

\section{Généralisation du problème}

\subsection{Changement de vitesse}

\begin{frame}
  Ce modèle ne prend pas en compte le changement de vitesse

  \begin{figure}
    \centering
    \begin{tikzpicture}
      \draw[help lines] (0,0) grid (5,5);
      \draw[<->] (0,5.5) -- (0,0) -- (5.5,0);
      \draw[thick,blue] (0,0) -- (2,3) -- (2.7,3) -- (5,4);
      \draw (0,6) node {charge de travail};
      \draw (6,0) node {temps};
      \draw[blue] (0.5,1.5) node {$v_1$};
      \draw[blue] (3.5,3.7) node {$v_2$};
      \draw[|-|,red,thick] (2,3.1) -- (2.7, 3.1);
      \draw[red] (2.35,3.5) node {$\tau$};
    \end{tikzpicture}
    \caption{Changement de vitesse}
    \label{fig:chg_speed}
  \end{figure}
\end{frame}

\begin{frame}
  \begin{figure}
    \begin{tikzpicture}
      \draw (2.5, 5.5) node {$v1 > v_2$};
      \draw[help lines] (0,0) grid (5,5);
      \draw[<->] (0,5.5) -- (0,0) -- (5.5,0);
      \draw[blue,thick] (1,0) -- (2.15,3.45);
      \draw[red,ultra thick] (2.15,3.45) -- (2.45, 3.45);
      \draw[red] (2.30, 3.6) node {$\tau$};
      \draw[blue,thick](2.45, 3.45) -- (4,5);
      \draw[green, thick,dashed] (1,0) -- (2,3) -- (4,5);
      \draw (2,-0.25) node {$t$};
      \draw (3,-0.25) node {$t+1$};
      \draw[purple,dashed,thick] (2,5) -- (2,0);
      \draw[purple,dashed,thick] (2.15,5) -- (2.15,0);
      \draw[red] (2.5,1.5) node {$\delta$};
      \draw[red,ultra thick] (2.0,1.5) -- (2.15,1.5);
    \end{tikzpicture}
  \end{figure}
\end{frame}

\begin{frame}
  \begin{figure}

    \begin{tikzpicture}
      \draw (2.5, 5.5) node {$v1 < v_2$};
      \draw[help lines] (0,0) grid (5,5);
      \draw[<->] (0,5.5) -- (0,0) -- (5.5,0);
      \draw[blue,thick] (1,0) -- (2.55,1.55);
      \draw[red,ultra thick] (2.55,1.55) -- (2.85,1.55);
      \draw[blue,thick] (2.85,1.55) -- (4,5);
      \draw[green, thick,dashed] (1,0) -- (3,2) -- (4,5);
      \draw[red] (2.70, 1.40) node {$\tau$};
      \draw (3,-0.25) node {$t$};
      \draw (4,-0.25) node {$t+1$};
      \draw[purple,dashed,thick] (3,5) -- (3,0);
    \end{tikzpicture}
  \end{figure}
\end{frame}

\subsection{Surcoût}

\begin{frame}{Surcoût}
  $c(\cdot)$, la fonction de consommation est convexe. La situation
  précédente entraîne donc un surcoût.

  \begin{equation}
    \label{eq:overcost}
    f(v_1,v_2)=
    \begin{cases}
      0 & \mbox{if } v_1=v_2 \\
      \delta\times c(v_1) - (\tau + \delta)\times c(v_2)+\tau\times c(0) & \mbox{si }
      v_1>v_2 \\
      \delta\times c(v_2) - (\tau + \delta)\times c(v_1)+\tau\times c(0) & \mbox{si }
      v_1<v_2
    \end{cases}
  \end{equation}
\end{frame}

\subsection{Politique}

\begin{frame}
  \begin{block}{Politique}
    $\pi$ est une politique et $\pi_t^v(w)\in\V$ est la vitesse
    choisie lorsque l'on est à l'état $w$, à l'instant $t$ et que la
    vitesse à $t-1$ était $v$
  \end{block}

  \begin{block}{$J_t^v(w)$}
    $J_t^v(w)$ est la consommation moyenne lorsque l'on suit la
    politique $pi$, que l'on est à l'état $w$, à l'instant $t$ et que
    la vitesse à $t-1$ était $v$
  \end{block}
  
\end{frame}

\subsection{Nouvel algorithme}

\begin{frame}{Nouvel algorithme}
  \begin{algorithm}[H]
    \caption{Algorithme dynamique de recherche d'une politique optimale
    prenant en compte le coût du changement de vitesse}
    \label{alg:chgspdalgo}  
    \begin{algorithmic}
      \STATE $t\leftarrow T$
      \FORALL{$w\in\W,v\in\V$}
      \STATE $J_T^v(w)\leftarrow0$
      \ENDFOR
      \WHILE{$t\geq1$}
      \FORALL{$w\in\W,v\in\V$}
      \STATE $J_{t-1}^v(w)\leftarrow \min\limits_{u\in\V,u\geq w(1)}
      c(u)+f(v,u)+\sum\limits_{w'\in\W}J_t^u(w')\times P_u(w,w')$
      \STATE $\pi_{t-1}^v(w)\leftarrow \argmin\limits_{u\in\V,u\geq w(1)}
      c(u)+f(v,u)+\sum\limits_{w'\in\W}J_t^u(w')\times P_u(w,w')$    
      \ENDFOR
      \STATE $t\leftarrow t-1$
      \ENDWHILE
      \RETURN $\pi^*$
    \end{algorithmic}
  \end{algorithm}
\end{frame}

\subsection{Résultats}

\begin{frame}
  \begin{figure}
    \centering
    \includegraphics[scale=0.4]{example_compare.png}
    \caption{Comparison of the work done in standard case (orange line) and the one
      with over cost (blue line). $T=30,\Delta=4,S=3$}
    \label{fig:algo_overcost}
  \end{figure}
\end{frame}

\subsection{Processeur travaillant à l'infini}

\begin{frame}
  \begin{algorithm}[H]
    \caption{Programmation dynamique pour trouver une politique stationnaire}
    \label{alg:inftyth}  
    \begin{algorithmic}
      \STATE $u^0\leftarrow (0,0,\dots,0),~u^1\leftarrow(1,0,\dots,0)$
      \STATE $n\leftarrow1$
      \STATE $\epsilon>0$
      \WHILE{$span(u^n(w)-u^{n-1}(w))\geq\epsilon$}
      \FORALL{$w\in\W$}
      \STATE $u^{n+1}(w)\leftarrow\min\limits_{v\in\V,v\geq
        w(1)}\left(c(v)+\sum\limits_{w'\in\W}P_v(w,w')\times u^n(w')\right)$
      \ENDFOR
      \STATE $n\leftarrow n+1$
      \ENDWHILE
      \STATE Choose any $w\in\W$ and let $g^*\rightarrow
      u^n(w)-u^{n-1}(w)$
      \FORALL{$w\in\W$}
      \STATE
      $\pi^*(w)\in\argmin\limits_{v\in\V,v\geq
        w(1)}\left(c(v)+\sum\limits_{w'\in\W}P_v(w,w')\times
        u^n(w')\right)$
      \ENDFOR
      \RETURN $\pi^*(\cdot)$
    \end{algorithmic}
  \end{algorithm}
\end{frame}

\section{Conclusion}

\begin{frame}{Conclusion}
  Nous avons vu
  \begin{itemize}
  \item un modèle de processeur effectuant des travaux
  \item comment optimiser la consommation d'énergie avec ce modèle
  \item nous avons amélioré ce modèle en prenant en compte le coût du
    changement de vitesse
  \item trouvé un algorithme pour ce nouveau modèle
  \item vu un modèle sans temps imparti
  \end{itemize}
\end{frame}

\end{document}
